% Voor Engelse tekst, gebruik de optie english achter documentclass.
\documentclass{uvamath}
\usepackage[dutch]{babel}

\usepackage{graphicx}
\usepackage[pdfborder={0 0 0}]{hyperref}
\usepackage{lipsum}

\title{Lossless Image Coding}
\author[henk@science.uva.nl, 6127901]{Henk de Vries}
%\author[ingrid@science.uva.nl, 6123102]{Ingrid de Vries}
%\author[ingrid@science.uva.nl, 6123102]{Ingrid de Vries}

\what{Projectverslag jaar 2} % voorbeeld
\supervisors{prof.\ dr.\ Carl Friedrich Gauss}
\secondgrader{dr.\ Karl Weierstrass}

\coverimage{\rule{9cm}{9cm}\\\emph{Voeg een plaatje in.}}
%\coverimage{\includegraphics[scale=0.5]{bestandsnaam}}


\begin{document}
\maketitle

\begin{abstract}
\lipsum[2-3]
\end{abstract}

\tableofcontents

\chapter{Inleiding}
\lipsum

\chapter{Theorie}
\lipsum[2]
\section{Basis}
\lipsum[5-6]
\section{Verder}
\lipsum[7]

\chapter{Resultaten}
\lipsum[2]

\chapter{Conclusie}
\lipsum[1]

\appendix

\chapter{Experimenten}
\lipsum[2]


\end{document}